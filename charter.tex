\documentclass[
11pt, % The default document font size, options: 10pt, 11pt, 12pt
codirector, % Uncomment to add a codirector to the title page
]{charter} 

% El títulos de la memoria, se usa en la carátula y se puede usar el cualquier lugar del documento con el comando \ttitle
\titulo{Título del proyecto} 

% Nombre del posgrado, se usa en la carátula y se puede usar el cualquier lugar del documento con el comando \degreename
\posgrado{Carrera de Especialización en Sistemas Embebidos} 
%\posgrado{Carrera de Especialización en Internet de las Cosas} 
%\posgrado{Carrera de Especialización en Intelegencia Artificial}
%\posgrado{Maestría en Sistemas Embebidos} 
%\posgrado{Maestría en Internet de las cosas}

% Tu nombre, se puede usar el cualquier lugar del documento con el comando \authorname
\autor{SISE-RS versión B} 

% El nombre del director y co-director, se puede usar el cualquier lugar del documento con el comando \supname y \cosupname y \pertesupname y \pertecosupname
\director{Nombre del Director}
\pertenenciaDirector{pertenencia} 
% FIXME:NO IMPLEMENTADO EL CODIRECTOR ni su pertenencia
\codirector{John Doe} % para que aparezca en la portada se debe descomentar la opción codirector en el documentclass
\pertenenciaCoDirector{FIUBA}

% Nombre del cliente, quien va a aprobar los resultados del proyecto, se puede usar con el comando \clientename y \empclientename
\cliente{Nombre del cliente}
\empresaCliente{Empresa del cliente}

% Nombre y pertenencia de los jurados, se pueden usar el cualquier lugar del documento con el comando \jurunoname, \jurdosname y \jurtresname y \perteunoname, \pertedosname y \pertetresname.
\juradoUno{Nombre y Apellido (1)}
\pertenenciaJurUno{pertenencia (1)} 
\juradoDos{Nombre y Apellido (2)}
\pertenenciaJurDos{pertenencia (2)}
\juradoTres{Nombre y Apellido (3)}
\pertenenciaJurTres{pertenencia (3)}
 
\fechaINICIO{30 de abril de 2021}		%Fecha de inicio de la cursada de GdP \fechaInicioName
\fechaFINALPlan{18 de junio de 2021} 	%Fecha de final de cursada de GdP
\fechaFINALTrabajo{15 de mayo de 2022}	%Fecha de defensa pública del trabajo final

\def\codigo{SISE-RS}
\newcommand{\req}[1]{\textbf{[\codigo-#1]:}}

\begin{document}

\maketitle
\thispagestyle{empty}
\pagebreak


\thispagestyle{empty}
{\setlength{\parskip}{0pt}
\tableofcontents{}
}
\pagebreak


\section*{Registros de cambios}
\label{sec:registro}

\begin{table}[ht]
\label{tab:registro}
\centering
\begin{tabularx}{\linewidth}{@{}|c|X|c|@{}}
\hline
\rowcolor[HTML]{C0C0C0} 
Revisión & \multicolumn{1}{c|}{\cellcolor[HTML]{C0C0C0}Detalles de los cambios realizados} & Fecha      \\ \hline
A & Creación del documento & 27/06/2021 \\ \hline
B & Se agrega encabezado en la plantilla del documento. \newline
	Modificación de la tabla de registro de cambios. \newline
	Nuevo formato de enumeración de requisitos.\newline
	Se amplia la sección \ref{sub:perspectiva}. & 03/07/2021 \\ \hline
\end{tabularx}
\end{table}

\pagebreak

\section{1. Introducción}
\label{sec:introduccion}

\subsection{Propósito}
\label{sub:proposito}

Este documento representa una especificación de requerimientos de software para un sistema de inyección de \emph{soft-errors} y un \emph{firmware} de \emph{self-testing}.
Está dirigido a las personas que se ocupen de las siguientes tareas:
\begin{itemize}
	\item análisis
	\item diseño
	\item implementación
	\item pruebas
\end{itemize}

\subsection{Ámbito del sistema}
\label{sub:ambito}

El nombre del sistema será SISE y permitirá inyectar errores en todos los registros accesibles del microcontrolador \emph{SAMV71}.
Su función será evaluar las técnicas de mitigación de \emph{soft-errors} en funciones a ser utilizadas en la misión \emph{Sabiamar}.
Adicionalmente, se proveerá un \emph{firmware} de \emph{self-testing} para determinar los presupuestos de \emph{hardware}.

El beneficio que se espera obtener es utilizar componentes que no fueron sometidos a un proceso de calificación (alternativos).
Además, se espera simular los 5 años de misión en un tiempo acelerado.

\subsection{Definiciones, acrónimos y abreviaturas}
\label{sub:definiciones}

\begin{enumerate}
	\item Definiciones:
		\begin{itemize}
			\item Sabiamar: constelación de dos satélites argentino-brasileños para la información del mar.
			\item soft-errors: modificación no destructiva del valor de un registro o memoria.
		\end{itemize}
	\item Acrónimos:
		\begin{itemize}
			\item CSV: comma separated value.
			\item IEEE: Institute of Electrical and Electronics Engineers.
			\item JTAG: Join Test Action Group.
			\item RAM: random access memory.
			\item TCP: transfer control protocol.
		\end{itemize}
	\item Abreviaturas:
		\begin{itemize}
			\item Std: estándar.
		\end{itemize}
\end{enumerate}

\subsection{Referencias}
\label{sub:referencias}
INVAP - Propuesta de tesis: sistema de inyección de soft-errors.

\subsection{Visión general del documento}
\label{sub:vision}

Este documento se realizó según lo especificado en el estándar IEEE Std. 830-1998.

\newpage
\section{2. Descripción general}
\label{seb:descripcion}

\subsection{Perspectiva del producto}
\label{sub:perspectiva}

El software aquí especificado es independiente de otros sistemas y no tiene relación con otros productos.

El principio de inyección de \emph{soft-erros} se basará en conectarse al microcontrolador a través de su interfaz de \emph{debug}; se procederá a suspender la ejecución del \emph{firmware} y luego se realizarán las modificaciones necesarias.

En la figura \ref{fig:esqInyeccion} se puede observar un esquema general del proceso de inyección de \emph{soft-errors}.

\begin{figure}[h]
	\centering
	\includegraphics[width=\textwidth]{./Figuras/inyeccion.png}
	\caption{Esquema general de inyección de \emph{soft-errors}.}
	\label{fig:esqInyeccion}
\end{figure}

El inyector de \emph{soft-errors} deberá modificar obtener la información de los registros del microcontrolador SAMV71 antes de realizar una modificación.
Finalmente, debe persistir la operación realizada junto a los datos previos a la inyección.

Adicionalmente, se entregará un \emph{firmware} de \emph{self-testing} que generará reportes sobre el funcionamiento de los periféricos del microcontrolador \emph{SAMV71}.
En la figura \ref{fig:esqSelfTesting} se puede ver un esquema general del proceso de \emph{self-testing}.

\begin{figure}[h]
	\centering
	\includegraphics[width=\textwidth]{./Figuras/selfTesting.png}
	\caption{Esquema general del proceso de \emph{self-testing}.}
	\label{fig:esqSelfTesting}
\end{figure}

\subsection{Funciones del producto}
\label{sub:funcionesProducto}

El software aquí especificado brindará las siguientes funcionalidades:

\begin{itemize}
	\item Inyección de errores en todos los registros accesibles del microcontrolador SAMV71.
	\item Monitoreo del estado de funcionamiento del microcontrolador SAMV71.
	\item Persistencia de los soft-errors inyectados.
	\item Persistencia de los informes de estado de funcionamiento.
	\item Permitir escribir ensayos de evaluación.
	\item Presentación de resultados en histogramas que permitan un análisis estadístico.
\end{itemize}

\subsection{Características de los usuarios}
\label{sub:usuarios}

Los usuarios finales de este producto son ingenieros de desarrollo del INVAP.

\subsection{Restricciones}
\label{sub:restricciones}

Las restricciones del desarrollo del sistema son las siguientes:

\begin{itemize}
	\item Utilización de repositorio con control de versiones \emph{Gitlab}.
	\item Documentación del código con \emph{Doxygen}.
	\item Utilización exclusiva del lenguaje de programación \emph{Python 3}.
\end{itemize}

\subsection{Suposiciones y dependencias}
\label{sub:suposiciones}

La suposición principal es que se tendrá acceso irrestricto al microcontrolador \emph{SAMV71} antes del día 01/11/2021.

\subsection{Requisitos futuros}
\label{sub:futuro}

N/A

\newpage
\section{3. Requisitos específicos}
\label{sec:requisitos}


\subsection{Interfaces externas}
\label{sub:interfaces}

\begin{itemize}
	\item \req{001} se comunicará de forma bidireccional con \emph{OpenOCD} a través de \emph{TCP} o \emph{Telnet}.
	\item \req{002} se deberá recibir y capturar la información proveniente por puerto \emph{Serial} del microcontrolador \emph{SAMV71}.
\end{itemize}

\subsection{Funciones}
\label{sub:funciones}

\begin{enumerate}
	\item Inyección de \emph{soft-errors}:
	\begin{itemize}
		\item \req{003} sobrescribirá los valores en memoria \emph{RAM}.
		\item \req{004} invertirá el valor de un bit en la memoria \emph{RAM}.
		\item \req{005} sobrescribirá valores en los registros internos.
		\item \req{006} invertirá el valor de un bit dentro de un registro interno.
		\item \req{007} reportará el estado del microcontrolador antes de realizar una inyección de \emph{soft-errors}.
		\item \req{008} interpretará una descripción de ensayo escrita en \emph{Python 3} y ejecutará las inyecciones según le indique.
		\item \req{009} aceptará descripciones de ensayo que impongan una tasa de error para cada registro interno.
		\item \req{010} procesará descripciones de ensayo que impongan una tasa de error para una posición o rango de memoria \emph{RAM}.
		\item \req{011} las descripciones de ensayo podrán especificar probabilidades de error con una resolución de 1 bit.
	\end{itemize}
	\item \emph{Firmware} de \emph{self-testing}:
	\begin{itemize}
		\item \req{012} detectará el funcionamiento anormal de los periféricos del microcontrolador \emph{SAMV71}.
		\item \req{013} reportará periódicamente el estado de los periféricos utilizando el protocolo \emph{Serial}.
		\item \req{014} los reportes tendrán un formato \emph{CSV} y un tamaño menor a 5 kB.
	\end{itemize}
	\item Almacenamiento de reportes:
	\begin{itemize}
		\item \req{015} se incluirá un \emph{timestamp} de recepción del reporte.
		\item \req{016} se generarán histogramas para su análisis estadístico.
	\end{itemize}
\end{enumerate}

\subsection{Requisitos de rendimiento}
\label{sub:rendimiento}

\req{017} el inyector de soft-errors deberá realizar la inserción solicitada en un tiempo menor a 10 ms.

\subsection{Restricciones de diseño}
\label{sub:restriccionesDiseño}

\req{018} se utilizará el microcontrolador \emph{SAMV71} como dispositivo principal.

\subsection{Atributos del sistema}
\label{sub:atributos}

\begin{enumerate}
	\item Mantenibilidad:
	\begin{itemize}
		\item \req{019} el software deberá permitir su modificación para trabajar con otras arquitecturas.
	\end{itemize}
\end{enumerate}

\subsection{Otros requisitos}
\label{sub:otros}

N/A.

\section{4. Apéndices}
\label{sec:apendices}

%\subsection{Formatos de entrada/salida}¸

%\subsection{Resultados de análisis de costes}

\subsection{Restricciones acerca del lenguaje de programación}

El lenguaje de programación será \emph{Python 3} y el código deberá ser documentado según las recomendaciones del manual de usuario de \emph{Doxygen}.

\end{document}
